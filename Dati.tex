%----------------------------------------------------------------------------------------
%   USEFUL COMMANDS
%----------------------------------------------------------------------------------------

\newcommand{\dipartimento}{Dipartimento di Matematica ``Tullio Levi-Civita''}

%----------------------------------------------------------------------------------------
% 	USER DATA
%----------------------------------------------------------------------------------------

% Data di approvazione del piano da parte del tutor interno; nel formato GG Mese AAAA
% compilare inserendo al posto di GG 2 cifre per il giorno, e al posto di
% AAAA 4 cifre per l'anno
\newcommand{\dataApprovazione}{Data}

% Dati dello Studente
\newcommand{\nomeStudente}{Marco}
\newcommand{\cognomeStudente}{Mazzucato}
\newcommand{\matricolaStudente}{1193113}
\newcommand{\emailStudente}{marco.mazzucato.4@studenti.unipd.it}
\newcommand{\telStudente}{+ 39 347 427 7677}

% Dati del Tutor Aziendale
\newcommand{\nomeTutorAziendale}{Luca}
\newcommand{\cognomeTutorAziendale}{Bizzaro}
\newcommand{\emailTutorAziendale}{luca.b@riskapp.it}
\newcommand{\telTutorAziendale}{+ 39 028 089 7581}
\newcommand{\ruoloTutorAziendale}{}

% Dati dell'Azienda
\newcommand{\ragioneSocAzienda}{RiskApp S.r.l}
\newcommand{\indirizzoAzienda}{Via Martiri della Libertà 19, Conselve (PD)}
\newcommand{\sitoAzienda}{https://www.riskapp.it/}
\newcommand{\emailAzienda}{info@riskapp.it}
\newcommand{\partitaIVAAzienda}{P.IVA IT04914960283}

% Dati del Tutor Interno (Docente)
\newcommand{\titoloTutorInterno}{Prof.}
\newcommand{\nomeTutorInterno}{Paolo}
\newcommand{\cognomeTutorInterno}{Baldan}

\newcommand{\prospettoSettimanale}{
     % Personalizzare indicando in lista, i vari task settimana per settimana
     % sostituire a XX il totale ore della settimana
    \begin{itemize}
        \item \textbf{Prima Settimana (40 ore)}
        \begin{itemize}
            \item Incontro con persone coinvolte nel progetto per discutere le richieste
            relativamente al sistema da sviluppare;
            \item Verifica credenziali e strumenti di lavoro assegnati;
            \item Presa visione dell’infrastruttura esistente;
            \item Inizio della formazione sulle tecnologie adottate;
        \end{itemize}
        \item \textbf{Seconda Settimana - Formazione e analisi(40 ore)}
        \begin{itemize}
            \item Studio di ReactJS e Redux;
            \item Analisi dei requisiti del prodotto;
        \end{itemize}
        \item \textbf{Terza Settimana - Formazione e analisi (40 ore)}
        \begin{itemize}
            \item Studio di ReactJS e Redux;
            \item Analisi dei requisiti del prodotto;
            \item Studio delle librerie adottate;
        \end{itemize}
        \item \textbf{Quarta Settimana - Sviluppo (40 ore)}
        \begin{itemize}
            \item Inizio codifica del prodotto;
        \end{itemize}
        \item \textbf{Quinta Settimana - Sviluppo (40 ore)}
        \begin{itemize}
            \item Codifica del prodotto con relativi test;
        \end{itemize}
        \item \textbf{Sesta Settimana - Sviluppo (40 ore)}
        \begin{itemize}
            \item Codifica del prodotto con relativi test;
        \end{itemize}
        \item \textbf{Settima Settimana - Sviluppo (40 ore)}
        \begin{itemize}
            \item Codifica del prodotto con relativi test;
        \end{itemize}
        \item \textbf{Ottava Settimana - Conclusione (40 ore)}
        \begin{itemize}
            \item Collaudo finale;
            \item Stesura documentazione;
        \end{itemize}
    \end{itemize}
}

% Indicare il totale complessivo (deve essere compreso tra le 300 e le 320 ore)
\newcommand{\totaleOre}{}

\newcommand{\obiettiviObbligatori}{
	 \item \underline{\textit{O01}}: Creazione del frontend per la rappresentazione dello schema societario;
	 \item \underline{\textit{O02}}: Interazione con i dati da backend via API REST sullo schema societario;
}

\newcommand{\obiettiviDesiderabili}{
	 \item \underline{\textit{D01}}: Richiamare e rappresentare notizie riguardanti gli esponenti e le società collegate alla società analizzata, usando altri servizi API di Riskapp per la ricerca di notizie;
}

\newcommand{\obiettiviFacoltativi}{
	 \item \underline{\textit{F01}}: Test a livello di frontend;
}
